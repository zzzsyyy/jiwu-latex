\documentclass{jiwu}

\thetitle{使用逐差法测量纸张厚度}
% 姓名、学号、班级
\theauthor{XXX}{XXXXXXXX}{XXXXXX}
\thedate{X/X/X}

\begin{document}
\maketitle
\thispagestyle{firstpage}
\section{实验目的}
\begin{enumerate}
    \item 学习使用列表法、逐差法记录并处理数据
    \item 熟练掌握A类、B类不确定度,以及合成不确定度的计算
    \item 熟练掌握游标卡尺的使用以及读数
    \item 学习掌握有效数字修约及规则
\end{enumerate}
\section{实验器材}
\begin{enumerate}
    \item 游标卡尺 ($\Delta_{仪}=\SI{0.02}{mm}$)
    \item 书本 (初始50页)
\end{enumerate}
\section{实验原理}
    \subsection{逐差法}
    设自变量与因变量之间存在线性关系,并测得一组值 $(x_1,y_1),(x_2,y_2),\cdots,(x_k,y_k)$,其中$k$为偶数($k=2n$),将数据分为两组:

    \begin{tabular}{*{9}{c}}
        $x_1$&$x_2$&$\cdots$&$x_n$&;&$x_{n+1}$&$x_{n+2}$&$\cdots$&$x_{2n}$\\
        $y_1$&$y_2$&$\cdots$&$y_n$&;&$y_{n+1}$&$y_{n+2}$&$\cdots$&$y_{2n}$\\
    \end{tabular}

    用后一组的值与前一组的值对应相减(间隔$n$项逐差),并利用公式$y=ax+b$,得到:

    $$
        \begin{array}{l}
            b_1=(y_{n+1}-y_1)/(x_{n+1}-x_1)\\
            b_2=(y_{n+2}-y_2)/(x_{n+2}-x_2)\\
            \vdots\\
            b_n=(y_{2n}-y_n)/(x_{2n}-x_n)
        \end{array}
    $$\par

    取平均值\par

    $$
        \bar{b}=\dfrac{1}{n}\sum_{i=1}^n \dfrac{y_{n+i}-y_i}{x_{n+i}-x_i}
    $$
    
    通常自变量等间隔分布,我们有
    $$\bar{b}=\dfrac{\overline{\Delta_ny}}{\Delta_nx}$$

    带入线性表达式得到
    $$\bar{a}=\dfrac{1}{k}(\sum_{i=1}^ky_i-\bar{b}\sum_{i=1}^kx_i)$$

    $b$的不确定度为

    $$u(b)=\bar{b}\sqrt{\left[\dfrac{u(\Delta_nx)}{\Delta_nx}\right]^2+\left[\dfrac{u(\Delta_ny)}{\Delta_ny}\right]^2}$$

    其中
    $$
        \begin{array}{l}
            u(\Delta_ny)=\sqrt{u_a^2(\Delta_ny)+u_b^2(\Delta_ny)}\\
            u(\Delta_nx)=u_b(\Delta_nx)
        \end{array}
    $$

    在等精良度测量条件下
    $$
        u_a(\Delta_ny)=\sqrt{\dfrac{\sum_{i=1}^k(\Delta_ny_i-\overline{\Delta_ny})^2}{(n-1)n}}
    $$

    这里$n=k/2$,同样地,$k$为奇数时,我们只需设$k=2n-1$.
    \subsection{逐差法在本实验中的应用}
        在本实验中,自变量纸张数量与因变量纸张厚度呈线性关系,且由于张数等间隔增加,可以应用逐差法处理实验数据。

        在本实验中,待测定的单张纸厚度偏小,与游标卡尺最小分度值接近,单张纸测量会造成较大粗大误差。因此可以采用一次测量多张纸总厚度的形式,减小误差。
\section{实验内容及步骤}
\subsection{调整实验器材}
取出游标卡尺,检查是否可以正常使用,归零后是否存在较大误差。
\subsection{测量数据}
\begin{enumerate}[label=\arabic*)]
    \item 将测量分为8组,首先取50页纸,通过游标卡尺测量其厚度。
    \item 增加10页纸,用游标卡尺测量其厚度。
    \item 重复步骤2),直至8组数据,记录在表格中。
\end{enumerate}
\subsection{数据处理}
利用逐差法对数据进行处理分析,计算纸张厚度与不确定度。
分10组处理,自50页起每次增加10页纸
\section{实验记录与数据处理}
\subsection{数据记录与处理}
\begin{table}[hbtp]
    \centering
\begin{tabular}{*{9}{|c}|}
    \hline
    序号 $i$&1&2&3&4&5&6&7&8\\
    \hline
    纸张数 $N_i$&50&60&70&80&90&100&110&120\\
    \hline
    厚度 $h_i$/\si{mm}&2.76&3.24&3.70&4.18&4.64&5.06&5.50&6.02\\
    \hline
\end{tabular}
\caption{原始数据}
    \begin{tabular}{*{5}{|c}|}
        \hline
        序号 i&1&2&3&4\\
        \hline
        $x_{n+i}-x_i$&40&40&40&40\\
        \hline
        $y_{n+i}-y_i$/\si{mm}&1.88&1.82&1.80&1.84\\
        \hline
    \end{tabular}
    \caption{逐差}
\end{table}

$\overline{\delta d}=\SI{1.835}{mm}$
\subsection{计算不确定度及结果}
$\Delta d$ 的A类不确定度$u_a(\Delta d)=\dfrac{1}{n}\sqrt{\dfrac{\dfrac{\sum\delta d^2}{n}-\overline{\delta d}^2}{n-1}}=\num{0.0042696}$,\medskip

B类不确定度$u_b(\Delta d)=\dfrac{1}{n}\dfrac{\Delta_\text{仪}}{\sqrt{3}}=\num{0.0028868}$,\medskip

合成不确定度$U(\Delta d)=\sqrt{u_a(\Delta d)^2+u_b(\Delta d)^2}=\num{0.0051539}$;\medskip

每页纸的厚度为$\dfrac{\Delta d}{40}=$ \SI{0.011469}{mm},不确定度为$\sqrt{(\dfrac{U(\Delta d)}{40})^2}=\num{0.00012885}$\medskip\bigskip

保留有效数字得最终结果,纸的厚度为 \SI{0.0115(0.0001)}{mm}
\end{document}
